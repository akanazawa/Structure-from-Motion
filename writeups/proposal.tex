\documentclass[a4paper]{article}
\usepackage{amsmath,hyperref}
\begin{document}
\title{Scientific Computing CS660 Fall '11 Final Project Proposal\\ Affine Structure from Motion: an Application of Singular Value Decomposition}
\author{Angjoo Kanazawa}
\date{\today}
\maketitle

\section{Introduction}
For this project I propose to explore an application of SVD in Computer
Vision domain. Structure from Motion (SfM) is the problem of
recovering 3D scene geometry and camera motion from a sequence of 2D
images. SfM has a wide range of application including 3D model
reconstruction of real-world objects, 3D motion matching for computer
graphics, virtual and augmented reality models, camera calibration and
many more.


SfM is a well studied problem with multitude of
approaches and problem statements. The project will focus on the
\emph{factorization} method proposed by Tomasi and Kanade, which recovers the structure and motion of
video sequences under the orthographic projection model. With this
model, the shape and motion can be recovered simultaneously using a
Singular Value Decomposition \cite{Tomasi}. This project will follow the project 4 of Derek Hoiem's CS 543/ECE 549
course at the University of Illinois at Urbana-Champaign:
\href{http://www.cs.illinois.edu/class/sp11/cs543/hw/hw4.pdf}{project
  description}. 


\section{Outline}
The outline of the project is as follows:
\begin{enumerate}
\item Description of the orthographic camera projection model
\item Assumptions and problem statement
\item Tomasi-Kanade Factorization method
\item Implementation of SfM: recovering a 3D point cloud of a short
  video sequence using the factorization method (using supplimental materials from the UIUC course web site)
\item Results and analysis
\end{enumerate}

\section{Overview of the Factorization Method}
This will be the main part of the project, but to demonstrate how
matrix factorization theorems we covered in class will be used, here
is an overview of the Tomasi-Kanade Factorization Method:

\subsection{Affine Cameras}
Projective geometry illustrates relationship between a single 2D image
point and it's corresponding 3D world point as (written in
homogeneous coordinates for numerical stability reasons): 
\begin{align*}
\begin{pmatrix}
  fx\\fy\\w
\end{pmatrix} &= \begin{pmatrix}
  f&0 & 0& 0 \\0 & f& 0 & 0 \\ 0 & 0 & 1 & 0 
\end{pmatrix}
\begin{pmatrix}
  X\\Y\\Z\\1
\end{pmatrix} \\
x &= PX
\end{align*}
Where $f$ is the focal length of the camera, $P$ is the projection
matrix of the camera, and the 2D location of the point in inhomogeneous coordinate is
$(fx/w, fy/w)$. For SfM, given $m$ images of $n$ fixed points, we have the equation $$x_{ij} = P_iX_j
\text{ }i=1,\dots,m, j=1,\dots,n.$$
 Orthography is a special case of perspective projection, where the 3D
world points are projected in parallel onto the image plane i.e. the
distance from the center of projection to the image is infinite and
$Z$ has no influence. 
i.e. we have the equation
$$\begin{pmatrix}
  x\\y\\1
\end{pmatrix}= \begin{pmatrix}
  1&0 & 0& 0 \\0 & 1& 0 & 0 \\ 0 & 0 & 0 & 1
\end{pmatrix}
\begin{pmatrix}
  X\\Y\\Z\\1
\end{pmatrix}$$


\subsection{Factorization Method}
Affine cameras combine the effect of affine transformation in the 3D
space, orthographic projection, and an affine transformation in the 2D
image space. i.e. $$P=[\text{3 by 3 affine transformation}]\begin{pmatrix}
  1&0 & 0& 0 \\0 & 1& 0 & 0 \\ 0 & 0 & 0 & 1
\end{pmatrix}[\text{4 by 4 affine transformation}] $$ For factorization, we can explicitly denote the mapping
and translation of the projection matrix $P$, and write it
as a linear combination of mapping and translation. In
inhomogeneous coordinates, this projection is $x =RX + t$, where $t$
is the translation and $R$ is the rotation/orientation of the camera.

For SfM, the image sequence is represented by a $2F\times P$ \emph{measurement matrix
}$W$, where $w_{fp} = (x_{fp}, y_{fp})^T$, and
$P$ is the number of points tracked through $F$
frames.  

To get rid of the translation term, we center the image points by
subtracting the mean (centroid) of image points, and assume that
the world coordinate system is at the centroid of the 3D points. Then,
the orientation (rotation)
of the camera at frame $f$ is represented by orthonormal
vectors $i_f, j_f, k_f \in \mathbf{R}^{3}$, where each vector
corresponds to the x, y, and z-axis of the image plane
respectively. Under orthography with the $z$ axis along the optical
axis, these vectors over $F$ frames are collected into a
\emph{motion matrix} $M\in \mathbf{R}^{2F \times 3}$ $$M =
\begin{pmatrix}
  i_1^T\\ \vdots \\  i_F^T \\ j_1^T \\ \vdots j_F^T
\end{pmatrix}
$$
We let $S_p = (X_p, Y_p, Z_p)^T$ be the 3D coordinates of feature $p$ in the fixed world point with the same origin. These vectors are collected into a
\emph{shape matrix} $S\in \mathbf{R}^{3 \times P}$ s.t. $S =
\begin{pmatrix}
  s_1 & \cdots & s_p
\end{pmatrix}^T$. Using this notation, for a single frame we get
\begin{align*}
  \begin{pmatrix}
    x_{fp}\\y_{fp}
  \end{pmatrix} &=
  \begin{pmatrix}
    i_{f}^T\\j_{f}^T
  \end{pmatrix}
  \begin{pmatrix}
    X_p\\ Y_p\\ Z_p
  \end{pmatrix}\\
w_{fp} &= M_fS_p  
\end{align*}

So for all frames, we have the equation $$W = MS$$
Our goal is to estimate $\hat M$ and $\hat S$, s.t. $\hat W = \hat M
\hat S$, our estimated
measurement matrix, and the actual $W$ is minimized i.e. $$\min_{M,
  S}||W- \hat M \hat S||^2$$
a least squares problem.
\cite{Tomasi} proved that under this model, the rank of $W$ is 3. So
we can achieve the least squares approximation by factoring $W$ by
SVD. Namely, 
\begin{align*}
W &= U\Sigma V^T\\
&\approx U_3\tilde \Sigma V_3^T \text{ because $W$ is rank 3}\\
&=
\begin{pmatrix}
  u_{1,1} & u_{1,2} & u_{1,3} \\ \vdots & & \vdots
\\ \vdots & & \vdots
\\   u_{2F,1} & u_{2F,2} & u_{2F,3} 
\end{pmatrix}
\begin{pmatrix}
  \sigma_1 & 0 & 0 \\
  0 & \sigma_2 & 0 \\
  0 & 0 & \sigma_3 \\
\end{pmatrix}
\begin{pmatrix}
  v_{1,1} &  \cdots &\cdots & v_{1,p}\\
  v_{2,1} &  \cdots &\cdots & v_{2,p}\\
  v_{3,1} &  \cdots &\cdots & v_{3,p}\\
\end{pmatrix}
\end{align*}

Where possible solution is to choose $\hat M = U_3\tilde \Sigma^{1/2}$
and $\hat S =\tilde \Sigma^{1/2}V_3^T$. This solution can be refined
by eliminating affine ambiguity, but this algorithm is numerically
stable and it is guaranteed to converge to the global minimum of the
least squares problem.
\bibliographystyle{plain}
\bibliography{reference}
\end{document}
