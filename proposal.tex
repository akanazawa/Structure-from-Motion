\documentclass[a4paper]{article}
\usepackage{amsmath,hyperref}
\begin{document}
\title{Scientific Computing CS660 Fall '11 Final Project Proposal\\ Affine Structure from Motion: an Application of Singular Value Decomposition}
\author{Angjoo Kanazawa}
\date{\today}
\maketitle

\section{Introduction}
For this project I propose to explore an application of SVD in Computer
Vision domain. Structure from Motion (SfM) is the problem of
recovering 3D scene geometry and camera motion from a sequence of 2D
images. SfM has a wide range of application including 3D model
reconstruction of real-world objects, 3D motion matching for computer
graphics, virtual and augmented reality models, camera calibration and
many more.


SfM is a well studied problem with multitude of
approaches and problem statements. The project will focus on the
\emph{factorization} method proposed by Tomasi and Kanade, which recovers the structure and motion of
video sequences under the orthographic projection model. With this
model, the shape and motion can be recovered simulatenously using a
Singular Value Decomposition \cite{Tomasi}. This project will follow the project 4 of Derek Hoiem's CS 543/ECE 549
course at the University of Illinois at Urbana-Champaign:
\href{http://www.cs.illinois.edu/class/sp11/cs543/hw/hw4.pdf}{project
  description}. 


\section{Outline}
The outline of the project is as follows:
\begin{enumerate}
\item Description of the orthographic camera projection model
\item Assumptions and problem statement
\item Tomasi-Kanade Factorization method
\item Implementation of SfM: recovering a 3D point cloud of a short
  video sequence using the factorization method (using supplimental materials from the UIUC course web site)
\item Results and analysis
\end{enumerate}

\section{Overview of the Factorization Method}
This will be the main part of the project, but to demonstrate how
matrix factorization theorems we covered in class will be used, here
is an overview of the Tomasi-Kanade Factorization Method:

Image sequence, the input to the algorithm, is represented by a $2F\times P$ \emph{measurement matrix
}$W$, where  $P$ is the number of points tracked through $F$
frames. Under orthography, given that the origin of the world
coordinates is set at the centroid of all $P$ points, the orientation
of the camera at frame $f$ is represented by a pair of orthonormal
vectors $i_f, j_f \in \mathbf{R}^{3}$ (dimension is 3 because we're
using homogenous coordinates, $\vec \hat x = (x,y,w)$, to avoid numerical instability for points
near infinity). These vectors over $F$ frames are collected into a
\emph{motion matrix} $M\in \mathbf{R}^{2F \times 3}$ $$M =
\begin{pmatrix}
  i_1^T\\ \vdots \\  i_F^T \\ j_1^T \\ \vdots j_F^T
\end{pmatrix}
$$




\bibliographystyle{plain}
\bibliography{reference}
\end{document}
